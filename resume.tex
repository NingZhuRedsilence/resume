\documentclass[12pt,letterpaper]{article}

\def\myauthor{Xunmo Yang}
\def\mytitle{XY-vita}
\def\myemail{xunmoyang@gmail.com}
%\def\myweb{pdgessler}
\def\myphone{832-519-8420}
\def\mykeywords{
  Xunmo Yang,
  resume,
  curriculum,
  vita,
  curriculum vita,
  cv,
  xunmo,
  pdg
}

\setlength\parindent{0pt}

\newenvironment{itemize*}%
{\begin{itemize}%
  \setlength{\itemsep}{0pt}}%
{\end{itemize}}

\usepackage[hyphens]{url}
\usepackage{fancyhdr}
\usepackage{lastpage}
%\usepackage{ocgtools}
\usepackage{enumitem}
\setlist{nolistsep,leftmargin=0.15in}
\usepackage[log-declarations=false]{xparse}
\usepackage[adobe-garamond]{mathdesign}
\usepackage[no-math]{fontspec}
\usepackage{microtype}
%\usepackage[margin=1.125in,top=1.375in,right=1in,left=2in]{geometry}
\usepackage[margin=1.2in,top=0.8in,right=0.7in,left=1.5in, bottom=1.3in]{geometry}
\usepackage[strict]{changepage}
\usepackage{fontawesome}
\usepackage[
  ocgcolorlinks,
  urlcolor={[rgb]{0,0,0.54}},
  unicode,
  plainpages=false,
  pdfpagelabels,
  pdftitle={\mytitle},
  pdfauthor={\myauthor},
  pdfkeywords={\mykeywords},
  breaklinks=true
]{hyperref}
%\usepackage[anythingbreaks]{breakurl}

% fix ocgcolor link breaking; thanks due to Benjamin Lerner (http://goo.gl/VZKR7M)
\makeatletter
\AtBeginDocument{%
  \newlength{\temp@x}%
  \newlength{\temp@y}%
  \newlength{\temp@w}%
  \newlength{\temp@h}%
  \def\my@coords#1#2#3#4{%
    \setlength{\temp@x}{#1}%
    \setlength{\temp@y}{#2}%
    \setlength{\temp@w}{#3}%
    \setlength{\temp@h}{#4}%
    \adjustlengths{}%
    \my@pdfliteral{\strip@pt\temp@x\space\strip@pt\temp@y\space\strip@pt\temp@w\space\strip@pt\temp@h\space re}}%
  \ifpdf
    \typeout{In PDF mode}%
    \def\my@pdfliteral#1{\pdfliteral page{#1}}% I don't know why % this command...
    \def\adjustlengths{}%
  \fi
  \ifxetex
    \def\my@pdfliteral #1{\special{pdf: literal direct #1}}% isn't equivalent to this one
    \def\adjustlengths{\setlength{\temp@h}{-\temp@h}\addtolength{\temp@y}{1in}\addtolength{\temp@x}{-1in}}%
  \fi%
  \def\Hy@colorlink#1{%
    \begingroup
      \ifHy@ocgcolorlinks
        \def\Hy@ocgcolor{#1}%
        \my@pdfliteral{q}%
        \my@pdfliteral{7 Tr}% Set text mode to clipping-only
      \else
        \HyColor@UseColor#1%
      \fi
  }%
  \def\Hy@endcolorlink{%
    \ifHy@ocgcolorlinks%
      \my@pdfliteral{/OC/OCPrint BDC}%
      \my@coords{0pt}{0pt}{\pdfpagewidth}{\pdfpageheight}%
      \my@pdfliteral{F}% Fill clipping path (the url's text) with current color
      \my@pdfliteral{EMC/OC/OCView BDC}%
      \begingroup%
        \expandafter\HyColor@UseColor\Hy@ocgcolor%
        \my@coords{0pt}{0pt}{\pdfpagewidth}{\pdfpageheight}%
        \my@pdfliteral{F}% Fill clipping path (the url's text) with \Hy@ocgcolor
      \endgroup%
      \my@pdfliteral{EMC}%
      \my@pdfliteral{0 Tr}% Reset text to normal mode
      \my@pdfliteral{Q}%
    \fi
    \endgroup
  }%
}
\makeatother
% end fixes

\newcommand{\ML}{\textsc{Matlab}}
\newcommand{\Simu}{Simulink}
\newcommand{\MLS}{\ML{}/\Simu{}}
\newcommand{\apdl}{\textsc{APDL}}
\newcommand{\ansys}{\textsc{Ansys}}
\newcommand{\fluent}{\textsc{Fluent}}
\newcommand\CPP{C/C\ensuremath{+}\ensuremath{+}}
\newcommand{\Star}{\textsc{Star-CCM\ensuremath{+}}}

\newcommand{\mhead}[1]{\leavevmode\marginpar{\sffamily\footnotesize #1}}
\newcommand{\rdate}[1]{{\addfontfeature{Numbers=OldStyle} \hfill #1}}
\renewcommand{\date}[1]{{\addfontfeature{Numbers=OldStyle} #1}}
\renewcommand{\labelitemi}{-}

\setmainfont[
  Ligatures={TeX,Common},
  BoldFont={AGaramondPro-Semibold},
]{Adobe Garamond Pro}
\setsansfont[
  Ligatures={TeX,Common},
  Letters=SmallCaps,
  Color=660000,
]{Adobe Garamond Pro}
\setmonofont[Scale=0.85]{FontAwesome}

\makeatletter % fix for \hrulefill w/ mathdesign package
\def\hrulefill{\leavevmode\leaders \hrule height \rulethickness \hfill\kern\z@}
\makeatletter

\begin{document}\flushbottom
\pagestyle{fancy} \setlength\headwidth{6.9in}
\rhead{\textsc{Xunmo Yang - r\'{e}sum\'{e} - \thepage{} of \pageref*{LastPage}}} \cfoot{}
\thispagestyle{empty}
\begin{adjustwidth}{-1in}{}
%{\Huge
%  {\textsc{%
%    {\addfontfeature{Style=TitlingCaps}P}\kern-1.5ptaul
%    {\addfontfeature{Style=TitlingCaps}D}\kern-2pt.~%
%    {\addfontfeature{Style=TitlingCaps}G}essler}
%  }
%}

{\Huge
  {\textsc{%
    {\addfontfeature{Style=TitlingCaps}X}\kern-1.5pt unmo
%    {\addfontfeature{Style=TitlingCaps}D}\kern-2pt.~%
    {\addfontfeature{Style=TitlingCaps}Y}ang}
  }
}
\hfill\hfill\hfill
{
  \begin{minipage}[b]{1.8in}
    \flushleft
%    \footnotesize
    \small
%    14019 Cypress Meadows Drive \\
%    Houston, TX~~77047\\
    832-519-8420\\
    \href{mailto:\myemail}{\myemail} \\
  \end{minipage}
  \hfill
  \begin{minipage}[b]{2in}
    \flushright
%    \footnotesize
    \small
%    \href{tel:\myphone}{\myphone} \\ %\texttt{?}~
%    \href{mailto:\myemail}{\myemail} \\
    \href{https://www.github.com/tcya}{\faGithub~~https://github.com/tcya}\\
    \href{https://www.linkedin.com/in/xunmoyang}{\faLinkedin~~linkedin.com/in/xunmoyang}
  \end{minipage}
}\par
\hrulefill
\end{adjustwidth}
\reversemarginpar
\setlength\marginparwidth{0.85in}
\smallskip
\mhead{Education}%
%\textbf{Udacity} - Machine Learning nanodegree \hfill {\rdate{\textsc{February} 2016}} \newline
%\smallskip
%\textbf{Udacity} - Data Analyst Nanodegree \hfill {\rdate{\textsc{September} 2015}} \newline
%\smallskip
\textbf{University of Houston} - Houston, TX, US  \\
\emph{Ph.D, Chemistry (GPA: 3.6)} \rdate{Anticipated graduation: \textsc{May} 2016}
\begin{itemize*}
%  \item Cumulative GPA: 3.44/4.00; {Energy Systems Specialization}
  \item Dissertation: \emph{Ab initio} Calculations of Intramolecular Charge and Energy Transfer with Reduced Modes in Donor-bridge-acceptor Species
  \item Advisor: Dr. \href{http://www.chem.uh.edu/people/faculty/Bittner/honors.php}{Eric R. Bittner}
%  \item Coursework:\newline
%    \begin{tabular}{ll}
%      \addlinespace\toprule
%      Advanced Engineering Mathematics & Transport Phenomena \\
%      Approx.~Methods in Engineering Analysis & Advanced Fluid Mechanics \\
%      Intermediate Thermodynamics & Sustainable Engineering \\
%      Statistical Thermodynamics & Combustion \\
%      \bottomrule
%    \end{tabular}
\end{itemize*}
\smallskip
\textbf{Xiamen University} - Xiamen, Fujian, China  \\
\emph{Bachelor of Science, Chemistry; Mathematics Minor} \rdate{\textsc{July} 2009}
\begin{itemize*}
%  \item GPA: 3.4 in major, 3.0 overall; Mathematics Minor
  \item Thesis: Study of Weak Interaction and Aromatic Carbon Atom in DREIDING Force Field
%  \item Selected Coursework:\newline
%    \begin{tabular}{ll}
%      \addlinespace\toprule
%
%      \bottomrule
%    \end{tabular}
\end{itemize*}
\smallskip
\textbf{Udacity} - \href{https://www.udacity.com/course/machine-learning-engineer-nanodegree--nd009}{Machine Learning Engineer Nanodegree} \hfill {\rdate{\textsc{February} 2016}} \newline
\smallskip
\textbf{Udacity} - \href{https://www.udacity.com/course/data-analyst-nanodegree--nd002}{Data Analyst Nanodegree} \hfill {\rdate{\textsc{September} 2015}} 

%\medskip


\bigskip
%\mhead{Software \newline Proficiencies}%
\mhead{Skills}%
\textbf{Working knowledge}\newline%?comfortable using in a professional setting
Python, Mathematica, R, HTML/CSS, D3.js, Octave/MATLAB, SQL, TensorFlow, SAS, Vim, Linux, \LaTeX,
Q-Chem, Gaussian, various chemistry instruments
%various quantum chemistry software

\smallskip
%\medskip
\textbf{Basic knowledge}\newline%?would require some training/use to refresh my memory
JavaScript, MongoDB,  Hadoop, HBase, Pig, Hive, Spark, Splunk, FORTRAN, C, Haskell

\smallskip
\textbf{Languages}\\
Fluent in English, Chinese. Proficient in Taiwanese.

\bigskip
%\mhead{Project \newline Experience}%
%\textbf{Udacity} - \href{https://www.udacity.com/course/deep-learning--ud730}{Deep Learning} \hfill {\rdate{\textsc{February} 2016}} \newline
%\href{https://github.com/tcya/tensorflow/tree/master/tensorflow}{\emph{English Letter Recognition}}
%\begin{itemize*}
%  \item Trained a 6 layer convolutional neural network with 95\% accuracy on \href{http://yaroslavvb.blogspot.com/2011/09/notmnist-dataset.html}{notMNIST} dataset using TensorFlow
%\end{itemize*}
%\medskip
%\textbf{Udacity} - \href{https://profiles.udacity.com/u/xunmoyang}{Machine Learning Nanodegree} \hfill {\rdate{\textsc{January - February} 2016}} \newline
%\href{https://github.com/tcya/Boston-house-price-prediction}{\emph{Boston House Price Study}}
%\begin{itemize*}
%  \item Studied the \href{http://scikit-learn.org/stable/modules/generated/sklearn.datasets.load_boston.html}{Boston house-prices dataset} and fine tuned a decision tree for price prediction
%\end{itemize*}
%
%\smallskip
%\href{https://github.com/tcya/Student-intervention-system}{\emph{Student Intervention System}}
%\begin{itemize*}
%  \item Built a student intervention system using AdaBoost to predict students' success and failure
% \end{itemize*}
%
%\smallskip
%\href{https://github.com/tcya/Creating-Customer-Segments}{\emph{Creating Customer Segments}}
%\begin{itemize*}
%  \item Applied PCA and ICA to a customer dataset of a wholesale distributor to preprocess and understand purchasing behavior better
%  \item Used Gaussian mixture model to find customer segments for better A/B test on policy change
%\end{itemize*}
%
%\smallskip
%\href{https://github.com/tcya/Q-Learning-Smartcab}{\emph{Train a Smartcab to Drive}}
%\begin{itemize*}
%  \item Taught toy smartcab traffic laws and best routing strategy with Q-learning 
% \end{itemize*}
%
%\smallskip
%\href{https://github.com/tcya/Vervet-monkey-alarm-clustering}{\emph{Clustering of Vervet Monkey's Alarms}} 
%\begin{itemize*}
%  \item Verified the classical discovery of three types  of alarms in vervet monkey by hierarchical, k-means and partitioning around medoids (PAM)  clustering  
%\end{itemize*}
%
%\medskip
%
%\textbf{Udacity} - {Intro to Hadoop and MapReduce} \hfill {\rdate{\textsc{October} 2015}} \newline
%\href{https://github.com/tcya/udacity-forum-hadoop-analysis}{\emph{Forum Data Analysis}}
%\begin{itemize*}
%  \item Analyzed the posts on Udacity's forum using Hadoop MapReduce codes
%\end{itemize*}
%
%\medskip
%
%\textbf{Udacity} - \href{https://profiles.udacity.com/u/xunmoyang}{Data Analyst Nanodegree} \hfill {\rdate{\textsc{January - September} 2015}} \newline
%\href{https://github.com/tcya/PISA-visualization}{\emph{PISA Data Visualization}}
%\begin{itemize*}
%  \item Explored the relations between family possessions and student scores in the Programme for International Student Assessment (PISA) data using R and Python
%  \item Visualized the analysis with interactions using D3.js and dimple.js
%\end{itemize*}
%
%\smallskip
%\href{https://github.com/tcya/Identify-Enron-Corporate-Fraud}{\emph{Identifying Fraud from Enron Email}}
%\begin{itemize*}
%  \item Investigated the Enron email corpus data with decision tree, Gaussian naive Bayesian, and k-means clustering machine learning techniques
%\end{itemize*}
%
%\smallskip
%\href{https://github.com/tcya/what-affects-red-wine-quality}{\emph{Red Wine Study}}
%\begin{itemize*}
%  \item Modeled the influence of various chemicals to red wine quality on a wine \href{https://s3.amazonaws.com/udacity-hosted-downloads/ud651/wineQualityInfo.txt}{dataset} by linear regression with Lasso
%\end{itemize*}
%
%\smallskip
%\href{https://github.com/tcya/houston-mapdata-wrangling}{\emph{Houston Map Data Wrangling}}
%\begin{itemize*}
%  \item Cleaned the map data on \href{www.openstreetmap.org}{openstreetmap} of the great Houston area (file size > 500M)
%  \item Analyzed the cleaned data with MongoDB queries
%\end{itemize*}
%
%\smallskip
%\href{https://github.com/tcya/new-york-subway-data-analysis}{\emph{New York Subway Data Analysis}}
%\begin{itemize*}
%  \item Statistically tested the relation between the ridership of subway and weather in New York
%\end{itemize*}
%
%\smallskip
%\emph{A/B Testing}
%\begin{itemize*}
%  \item Evaluated a hypothetical A/B test trying to reduce the number of frustrated students after enrollment on Udacity
%\end{itemize*}
%
%\medskip
%
%\href{https://github.com/tcya/Increment_by_1_Turing_machine}{\emph{Increased-by-one Single Tape Turing Machine}}\hfill {\rdate{\textsc{June} 2015}} \begin{itemize*}
%  \item Implemented an increased-by-one single tape Turing machine program with only HTML/CSS, inspired by a \href{https://stackoverflow.com/questions/2497146/is-css-turing-complete}{discussion} of the Turing completeness of HTML/CSS
%\end{itemize*}

\mhead{Project \newline Experience}%
\href{https://github.com/tcya/tensorflow/tree/master/tensorflow}{English Letter Recognition} \hfill {\rdate{\textsc{February} 2016}}
\begin{itemize*}
  \item Trained a 6 layer convolutional neural network with 95\% accuracy on \href{http://yaroslavvb.blogspot.com/2011/09/notmnist-dataset.html}{notMNIST} dataset using TensorFlow
\end{itemize*}

\smallskip
\href{https://github.com/tcya/Vervet-monkey-alarm-clustering}{Clustering of Vervet Monkey's Alarms}\hfill {\rdate{\textsc{February} 2016}}
\begin{itemize*}
  \item Verified the classical discovery of three types  of alarms in vervet monkey by hierarchical, k-means and partitioning around medoids (PAM)  clustering, achieved at least 75\% classification accuracy
  \item Built an AdaBoost model with 99\% prediction accuracy
\end{itemize*}

\smallskip
\href{https://github.com/tcya/Creating-Customer-Segments}{Creating Customer Segments} \hfill {\rdate{\textsc{January} 2016}}
\begin{itemize*}
  \item Applied PCA and  independent component analysis (ICA) to a customer dataset of a wholesale distributor to preprocess and understand purchasing behavior better
  \item Used Gaussian mixture model to find customer segments for better A/B test on policy change
\end{itemize*}

\smallskip
\href{https://github.com/tcya/Q-Learning-Smartcab}{Train a Smartcab to Drive}\hfill {\rdate{\textsc{January} 2016}}
\begin{itemize*}
  \item Taught toy smartcab traffic laws and best routing strategy with Q-learning. The driving agent was able to consistently reach the destination within allotted time with 90\% success rate
 \end{itemize*}

\smallskip
\href{https://github.com/tcya/udacity-forum-hadoop-analysis}{Forum Data Analysis} \hfill {\rdate{\textsc{October} 2015}}
\begin{itemize*}
  \item Analyzed the posts on Udacity's forum using Hadoop MapReduce codes
\end{itemize*}

\smallskip
\href{https://github.com/tcya/PISA-visualization}{PISA Data Visualization} \hfill {\rdate{\textsc{September} 2015}}
\begin{itemize*}
  \item Explored the relations between family possessions and student scores in the Programme for International Student Assessment (PISA) data using R and Python
  \item Visualized the analysis with interactions using D3.js and dimple.js
\end{itemize*}

\smallskip
\href{https://github.com/tcya/what-affects-red-wine-quality}{Red Wine Study} \hfill {\rdate{\textsc{September} 2015}}
\begin{itemize*}
  \item Modeled the influence of various chemicals to red wine quality on a wine \href{https://s3.amazonaws.com/udacity-hosted-downloads/ud651/wineQualityInfo.txt}{dataset} by linear regression with Lasso in RStudio
\end{itemize*}

\smallskip
\href{https://github.com/tcya/Increment_by_1_Turing_machine}{Increased-by-one Single Tape Turing Machine}\hfill {\rdate{\textsc{June} 2015}} \begin{itemize*}
  \item Implemented an increased-by-one single tape Turing machine program with only HTML/CSS, inspired by a \href{https://stackoverflow.com/questions/2497146/is-css-turing-complete}{discussion} of the Turing completeness of HTML/CSS
\end{itemize*}

\smallskip
\href{https://github.com/tcya/Identify-Enron-Corporate-Fraud}{Identifying Fraud from Enron Email} \hfill {\rdate{\textsc{May} 2015}}
\begin{itemize*}
  \item Investigated the Enron email corpus data with decision tree, Gaussian naive Bayesian, and k-means clustering machine learning techniques
\end{itemize*}

\smallskip
\href{https://github.com/tcya/houston-mapdata-wrangling}{Houston Map Data Wrangling} \hfill {\rdate{\textsc{May} 2015}}
\begin{itemize*}
  \item Cleaned the map data on \href{www.openstreetmap.org}{openstreetmap} of the great Houston area (file size > 500M)
  \item Analyzed the cleaned data with MongoDB queries
\end{itemize*}

\smallskip
\href{https://github.com/tcya/new-york-subway-data-analysis}{New York Subway Data Analysis} \hfill {\rdate{\textsc{May} 2015}}
\begin{itemize*}
  \item Statistically tested the relation between the ridership of subway and weather in New York
\end{itemize*}


\bigskip

\mhead{Certificates}%
%\textbf{Udacity:} 6 computer science courses (certificates available on my LinkedIn) \\
%\textbf{edX:} 2 computer science courses (certificates available on my LinkedIn)\\
%\textbf{Coursera:} 2 computer science courses (certificates available on my LinkedIn)
\href{https://rawgit.com/tcya/tcya.github.io/master/assets/images/SASBaseCertificate.pdf}{SAS Certified Base Programmer for SAS 9 Credential}\\
11 computer science courses on edX, Coursera and Udacity (certificates available on my \href{https://www.linkedin.com/in/xunmoyang}{LinkedIn}) 

\bigskip

\mhead{Research \& \\ Teaching \newline Experience}%
%\textbf{Udacity}\newline
%\emph{Data Analyst Nanodegree Team Guide} \rdate{\textsc{October}~2015 - \textsc{present}}
%\begin{itemize*}
%  \item Help other students in the nanodegree program commit to their studies via online guidance
%\end{itemize*}
%
%\smallskip
%
\textbf{University of Houston,} Houston, TX \newline
\emph{Research} \& \emph{Teaching Assistant} \rdate{\textsc{August}~2010 - \textsc{present}}
\begin{itemize*}
  \item Developed and coded in Mathematica a new theoretical molecular dynamics analysis scheme based on Lanczos algorithm and time-convolutionless master equation
  \item Benchmarked the scheme with a classical series of molecules and researched the dynamics
  \item Optimized the geometry of tripodal amine-Cu(I) complexes using density functional theory (DFT), to assist further research of their reactivity and stability
%   against molecular oxygen oxidation and ligand dissociation/exchange.
%  \item Collaborate to maintain Linux servers in the lab
  \item Taught general and physical chemistry labs independently. Instruments used include UV/VIS, FT-IR, ESR, NMR, STM and XRD
\end{itemize*}

%\emph{Teaching Assistant} \rdate{\textsc{August}~2010 - \textsc{present}}
%\begin{itemize*}
%  \item Teach general and physical chemistry labs independently. Instruments used include UV/VIS, FT-IR, ESR, NMR, STM and XRD
%  \item Organized chemistry stockroom and inventoried hazardous chemicals according to MSDS
%\end{itemize*}


\smallskip
\textbf{Xiamen University,} Xiamen, Fujian, China \newline
\emph{Research Assistant} \rdate{\textsc{August}~2009 - \textsc{June} 2010}
\begin{itemize*}
  \item Implemented FORTRAN programs for the point group and atom type recognition in AMBER and DREIDING force fields, as part of efficient QM/MM method development
\end{itemize*}

\bigskip
\mhead{Publications}%
%\href{http://pubs.acs.org/doi/abs/10.1021/jp503041y}{Intramolecular Charge and Energy Transfer Rates with Reduced Modes: Comparison to Marcus Theory for Donor-Bridge-Acceptor Systems}\\
\href{http://pubs.acs.org/doi/abs/10.1021/jp503041y}{\textbf{Intramolecular Charge and Energy Transfer Rates with Reduced Modes: Comparison to Marcus}}\\
\href{http://pubs.acs.org/doi/abs/10.1021/jp503041y}{\textbf{Theory for Donor-Bridge-Acceptor Systems}}\\
Yang, Xunmo and Bittner, Eric. \emph{The Journal of Physical Chemistry A}, \textbf{2014}, \emph{118}(28), pp 5196-5203

\href{http://scitation.aip.org/content/aip/journal/jcp/142/24/10.1063/1.4923191}{\textbf{Computing Intramolecular Charge and Energy Transfer Rates using Optimal Modes}}\\
Yang, Xunmo and Bittner, Eric. \emph{The Journal of Chemical Physics, \emph{142}, 244114 (2015)}

\smallskip
\textbf{Tripodal Amine Ligands for Accelerating Cu-Catalyzed Azide-Alkyne Cycloaddition: Efficiency and Stability against Oxidation and Dissociation}\\
Zhiling Zhu, Siheng Li, Haoqing Chen, Yongkai Huang, Xunmo Yang, Eric Bittner, and Chengzhi Cai. (Submitted to \emph{Organic} \& \emph{Biomolecular Chemistry})


\bigskip
\mhead{Patent}%
\href{https://www.google.com/patents/CN2665845Y?cl=en}{Coriolis force experiment plate} \quad\quad\quad\quad No.: CN 2665845 Y  \quad\quad\quad\quad Issued: 12/22/2004

\end{document}
